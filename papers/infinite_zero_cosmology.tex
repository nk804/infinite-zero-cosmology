\documentclass[11pt,a4paper]{article}
\usepackage[utf8]{inputenc}
\usepackage{amsmath,amssymb}
\usepackage{graphicx}
\usepackage{hyperref}
\usepackage{geometry}
\geometry{margin=1in}

\title{Infinite Zero Cosmology: \\
A White-Hole Projection Framework for Dark Energy and Dark Matter}
\author{Nataliya Khomyak}
\date{October 2025}

\begin{document}

\maketitle

\begin{abstract}
We present the Infinite Zero cosmological framework, proposing that reality arises from the dynamic balance between opposite charges or curvatures. The zero state ($0/0$) is not ``nothing'' but a neutral equilibrium where equal positive and negative contributions coexist. When this equilibrium is locally perturbed through ``vacuum punctures,'' it releases energy into lower dimensions as structured matter and fields. This framework provides a unified explanation for dark energy (residual pressure from past white-hole projections), dark matter (incomplete or frozen projections), and various cosmological anomalies including bulk flows, lensing residuals, and fast radio bursts. We present testable observational signatures and falsifiable predictions.
\end{abstract}

\section{Core Premise}

Reality arises from the dynamic balance between opposite charges or curvatures. The zero state ($0/0$) is not ``nothing''; it is a neutral equilibrium where equal positive and negative contributions coexist. When that equilibrium is locally perturbed, it releases energy into lower dimensions as structured matter and fields.

\section{Theoretical Framework}

\subsection{Geometry of Creation}

At Planck scale, the vacuum behaves like quantum foam composed of microscopic strings or loops. A local symmetry break --- a ``puncture'' --- creates an asymmetric pressure between positive and negative curvature domains. From the perspective of a lower-dimensional observer, this appears as a white-hole-like event: information and energy radiating outward into spacetime.

\subsection{Continuous, Fractal Genesis}

Creation is not a one-time Big Bang but an ongoing process. Each white-hole projection seeds new structure; its complement, a black hole, re-absorbs information. Together they maintain global balance --- the yin and yang of information flow --- so total neutrality of the cosmos is preserved even as it evolves.

\subsection{Charge-Based Unification}

Matter, forces, and spacetime curvature are different manifestations of charge symmetry:
\begin{itemize}
\item $+1$ and $-1$ are active charges or curvatures
\item $0$ is their neutral combination, a stable ``infinite zero'' equilibrium
\end{itemize}

Quantum and gravitational interactions emerge from oscillations around this equilibrium.

\subsection{Dark Sector Interpretation}

\begin{itemize}
\item \textbf{Dark energy:} Residual outward pressure from past white-hole projections; the vacuum retains a gentle net expansion push
\item \textbf{Dark matter:} Incomplete or ``frozen'' projections --- regions where the field never relaxed back to perfect neutrality, producing mass without light
\end{itemize}

\subsection{Information Conservation}

Every emission (white-hole outflow) corresponds to an absorption (black-hole inflow) elsewhere in the network. Information is never lost; it circulates through the web of projections, maintaining the universe's computational integrity.

\section{Mathematical Formulation}

\subsection{Scalar Field Dynamics}

We use a scalar field $\phi$ with potential $V(\phi)$ defining local vacuum energy:
\begin{equation}
\Lambda_{\text{local}} = 8\pi G\,V(\phi)
\end{equation}

Localized deviations $\Delta\phi \rightarrow \Delta\Lambda(x)$ produce pressure gradients that manifest as dark-energy currents.

The evolution obeys the Klein-Gordon equation:
\begin{equation}
\Box\phi = \frac{dV}{d\phi}
\end{equation}

This gives a self-consistent description of creation and relaxation cycles.

\subsection{Vacuum Puncture Model}

A ``puncture'' is a region where $\phi$ is momentarily displaced from $\phi_0$ by $\Delta\phi$. That displacement changes $V(\phi)$ locally, hence changes the local vacuum pressure:
\begin{equation}
\Lambda_{\text{local}} = \Lambda + \Delta\Lambda(x) = 8\pi G\,V(\phi_0+\Delta\phi(x))
\end{equation}

The gradient $\nabla\Delta\Lambda$ acts like a pressure differential in the cosmic fluid.

\subsection{Energy Flow}

In general relativity, a varying $\Lambda$ behaves as an effective stress-energy tensor:
\begin{equation}
T^{\mu\nu}_{\text{vac}} = -\frac{\Delta\Lambda(x)}{8\pi G}\,g^{\mu\nu}
\end{equation}

The divergence gives the momentum exchange:
\begin{equation}
\nabla_\mu T^{\mu\nu}_{\text{vac}} = -\frac{1}{8\pi G}\,\partial^\nu\Delta\Lambda(x)
\end{equation}

This term represents the ``dark-energy current'' that can steer local expansion rates.

\subsection{Geometry of the Puncture}

For a spherically symmetric perturbation:
\begin{equation}
\Delta\phi(r) = \Delta\phi_0\,e^{-(r/r_0)^2}
\end{equation}

Then $\Delta\Lambda(r)$ falls off as $e^{-2(r/r_0)^2}$. At large distances, space returns to uniform $\Lambda$; near $r = 0$, there is a bubble of slightly higher or lower vacuum energy --- a white-hole-like region ejecting energy outward.

\section{Observable Consequences}

If valid, the model predicts:

\begin{enumerate}
\item \textbf{Subtle anisotropies in cosmic expansion} (bulk flows)
\item \textbf{Weak-lensing patterns} not explained by visible mass
\item \textbf{Small deviations in the cosmic microwave background's} integrated Sachs-Wolfe signal
\item \textbf{Fast Radio Burst clustering} correlated with puncture regions
\item \textbf{High-energy transients} without obvious progenitors
\item \textbf{Gravitational wave echoes} from non-standard black hole interiors
\end{enumerate}

These would trace the lingering footprints of past projection events.

\section{Testable Predictions}

\subsection{Bulk-Flow Anomalies}

\textbf{Hypothesis:} Localized outward pushes (punctures) create stream-like velocity fields.

\textbf{Signature:} Large-scale, coherent peculiar velocity fields not explained by visible mass distribution (non-Gaussian velocity correlations on $>100$ Mpc scales).

\textbf{Test:} Use peculiar-velocity surveys (Tully-Fisher, Cosmicflows) and compare to simulated puncture models; search for directional correlations with weak-lensing or CMB ISW anomalies.

\subsection{Weak-Lensing Residuals}

\textbf{Hypothesis:} ``Frozen'' or partially realized projection events produce localized mass-like contributions that couple gravitationally but not electromagnetically.

\textbf{Signature:} Lensing maps showing mass where baryons are absent, but with different small-scale structure (e.g., higher granularity, distinct radial profiles) than standard cold dark matter halos.

\textbf{Test:} Cross-compare weak-lensing reconstructions vs. galaxy counts with high resolution; search for small, compact lensing anomalies whose mass profiles differ from NFW cusps.

\subsection{CMB Anomalies and ISW Residuals}

\textbf{Hypothesis:} Early or overlapping projection fronts imprint late-time integrated Sachs-Wolfe (ISW) anisotropies and subtle non-Gaussian features.

\textbf{Signature:} Localized cold/hot spots or directional ISW excesses correlated with bulk-flow or lensing anomalies.

\textbf{Test:} Cross-correlate Planck/WMAP CMB residual maps with candidate puncture regions identified by peculiar velocity and lensing studies.

\subsection{Fast Radio Bursts}

\textbf{Hypothesis:} Sudden local reconfigurations of vacuum geometry (micro-punctures that briefly couple to baryonic plasma) produce transient electromagnetic releases.

\textbf{Signature:} FRBs with locations that show atypical clustering relative to galaxy types or alongside anomalous local peculiar velocities and lensing patches.

\textbf{Test:} Check FRB catalogs for statistical clustering relative to bulk-flow maps or void boundaries; investigate unusual DM/repetition properties near candidate puncture zones.

\subsection{High-Energy Transients}

\textbf{Hypothesis:} Micro-punctures that partially actualize can inject energetic quanta (not necessarily baryons) into the intergalactic medium.

\textbf{Signature:} High-energy photons or neutrinos with no plausible progenitor (no merger, no supernova remnant) appearing in coherent regions.

\textbf{Test:} Cross-match IceCube/LHAASO/Fermi events with lensing/velocity anomalies and with FRB positions.

\subsection{Gravitational-Wave Anomalies}

\textbf{Hypothesis:} Non-standard interior geometries (black holes with bounce/inner white-hole region) produce non-merger GW echoes or unusual post-merger signatures.

\textbf{Signature:} Echoes or late-time deviations in GW strain not consistent with standard ringdown.

\textbf{Test:} Search LIGO/Virgo/KAGRA event residuals for late echoes or tiny non-GR features correlated with candidate puncture regions.

\section{Falsification Criteria}

A good hypothesis is falsifiable. The following observations would challenge or disprove this model:

\begin{itemize}
\item Bulk-flow and lensing anomalies, when mapped with sufficient precision, are fully explained by mass distributions with no unexplained residuals correlated across datasets
\item FRBs and high-energy transients show unambiguous progenitors in every case (no orphan transients)
\item High-precision CMB ISW and weak-lensing cross-correlations place strong upper limits on regional $\Lambda(x)$ variations incompatible with the magnitude required by the model
\end{itemize}

\section{Research Program}

\subsection{Immediate Next Steps}

\begin{enumerate}
\item \textbf{Physical validation:} Test heat signature of neutral-state operations
\item \textbf{Toy scalar-field simulation:} Evolve $\phi(x,t)$ with potential $V(\phi)$ that allows small localized $\Delta\phi$ seeds; measure how $\Delta\Lambda(x)$ forms
\item \textbf{N-body runs with local $\Lambda(x)$:} Modify N-body code to include spatially varying cosmological constant
\item \textbf{Data comparisons:} Create mock catalogs and compare to Cosmicflows, DES/LSST/Euclid lensing, Planck CMB residuals, FRB catalogs, and IceCube neutrinos
\end{enumerate}

\subsection{Long-Term Directions}

\begin{itemize}
\item Lab analogues: Propose analogue gravity tests (BEC horizons, optical horizons, Casimir modulation)
\item Integration with quantum gravity frameworks
\item Biological analogues of charge-state systems
\end{itemize}

\section{Conclusion}

The Infinite Zero cosmology proposes that the universe is a self-balancing field of opposites. Zero is not emptiness but the generative medium from which all polarity --- and thus all existence --- emerges. White holes and black holes are complementary valves of that equilibrium, keeping the infinite neutral state dynamically alive.

This framework provides testable predictions for dark energy, dark matter, and various cosmological anomalies. The path forward requires collaboration between observational astronomers, theoretical physicists, and computational scientists to validate or falsify these predictions.

\textbf{A single principle --- neutrality disturbed --- may give rise to matter, energy, space, time, and the dark components that fill the cosmic ledger.}

\bibliographystyle{plain}

\end{document}
